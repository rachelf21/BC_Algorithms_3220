\documentclass[11pt]{article}
\usepackage{preamble}

\usepackage{listings, courier}
\usepackage{mdframed} %for background color
\lstset{basicstyle=\small\ttfamily,breaklines=true}
%mathtools included in preamble
\DeclarePairedDelimiter\ceil{\lceil}{\rceil}
\DeclarePairedDelimiter\floor{\lfloor}{\rfloor}


\titleformat*{\section}{\Large\bfseries}

\title{CISC 3220 Homework Chapter 9}
\author{Rachel Friedman}
\date{April 26, 2020}

\begin{document}
\maketitle

\section*{Exercise 9.1}\nointerlineskip
\noindent \rule{\linewidth}{0.01pt}\\
\subsubsection*{Question 9.1-1}\nointerlineskip

Show that the second smallest of $n$ elements can be found with $n+\lceil \text{log }n \rceil - 2 $ comparisons in the worst case.\\

The smallest element is found in $n-1$ comparisons, through a tournament style comparison where each one of the ``winners'' advances to the next round. This results in $\lceil \text{log }n \rceil$ rounds. The second smallest element is one of those $\lceil \text{log }n \rceil$ elements. We now make $\lceil \text{log }n \rceil -1$ comparisons with those elements. So, in total, the amount of comparisons is $n-1 + (\lceil \text{log }n \rceil - 1)$ which is equal to $n+\lceil \text{log } n \rceil - 2 $ comparisons.\\

\section*{Exercise 9.2}\nointerlineskip
\noindent \rule{\linewidth}{0.01pt}\\
\subsubsection*{Question 9.2-3}\nointerlineskip
Write an iterative version of RANDOMIZED-SELECT.\\

\begin{center}  
\begin{multicols}{2}
\begin{mdframed}[backgroundcolor=gray!5]
\begin{lstlisting}
PARTITION(A, p, r)
    x = A[r]
    i = p
    for k = p - 1 to r
       if A[k] < x
           i = i + 1
           swap A[i] with A[k]
    i = i + 1
    swap A[i] with A[r]
    return i
\end{lstlisting}
\end{mdframed}    

\begin{mdframed}[backgroundcolor=gray!5]
\begin{lstlisting}    
RANDOMIZED-PARTITION(A, p, r)
    x = RANDOM(p - 1, r)
    swap A[x] with A[r]
    return PARTITION(A, p, r)
\end{lstlisting}
\end{mdframed}
\columnbreak

\begin{mdframed}[backgroundcolor=gray!5]

\begin{lstlisting}
RANDOMIZED-SELECT(A, p, r, i)
    while true
        if p == r
            return A[p]
        q = RANDOMIZED-PARTITION(A, p, r)
        k = q - p + 1
        if i == k
            return A[q]
        if i < k
            r = q
        else
            p = q
            i = i - k
    
\end{lstlisting}
\end{mdframed}
\end{multicols}

\end{center}

\newpage
\section*{Exercise 9.3}\nointerlineskip
\noindent \rule{\linewidth}{0.01pt}\\
\subsubsection*{Question 9.3-7}\nointerlineskip
Describe an $\mathcal{O}(n)$ running time algorithm that, given a set $S$ of $n$ distinct numbers and a positive integer $k \leq n$, determines the $k$ numbers in $S$ that are closest to the median of $S$.\\

For simplicity's sake, assume $n$ is odd and $k$ is even and the set $S$ is sorted:
\begin{enumerate}
\item Use linear time selection to find the median in position $n/2$.
\item Use linear time selection to find the element in position $(n-k)/2$.
\item Use linear time selection to find the element in position $(n+k)/2$.
\item Then traverse the set $S$ to find the elements that are less than $(n-k)/2$, and greater than $(n+k)/2$, and not equal to $n/2$.
\end{enumerate}
Thus, the algorithm takes $\mathcal{O}(n)$ times, since we use linear time selection exactly three times, and traverse the set once.\\

\subsubsection*{Question 9.3-8}\nointerlineskip
Let $X \lceil 1 ..n \rceil$ and $Y \lceil 1 ..n \rceil$ be two arrays, each containing $n$ numbers already in sorted order. Give an $\mathcal{O}(\log n)$-time algorithm to find the median of all 2$n$ elements in arrays $X$ and $Y$.\\


Repeat the following until length of array is 1:
\begin{enumerate}
\item Get the median of each array.
\item Take the upper portion of the array with the lower median.
\item Take the lower portion of the array with the higher median.
\item Is there more than one element left in each array?
\begin{enumerate}
\item Yes - go back to step 1.
\item No - return the lower number of the two.\\
\end{enumerate}
\end{enumerate}

\begin{center}  
\begin{minipage}{5in}
\begin{lstlisting}[language=Python]
def find_median_two_arrays(a, b):
    if len(a) == 1:
        return min(a[0], b[0])

    median = len(a)/2
    i = median + 1
    if a[median] < b[median]:
        return find_median_two_arrays(a[-i:], b[:i])
    else:
        return find_median_two_arrays(a[:i], b[-i:])
        
\end{lstlisting}
\end{minipage}
\end{center}

\vspace{20pt}





\end{document}