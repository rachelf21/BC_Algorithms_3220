\documentclass[11pt]{article}
\usepackage{preamble}
\titleformat*{\section}{\Large\bfseries}

\title{CISC 3220 Homework Chapter 3}
\author{Rachel Friedman}
\date{February 16, 2020}

\begin{document}
\maketitle

\section*{Problem 3-4}\nointerlineskip
\noindent \rule{\linewidth}{0.01pt}
Let $f(n)$ and $g(n)$ be asymptotically positive functions. Prove or disprove each of the following conjectures.\\

\subsubsection*{Question a}

$f(n) = \mathcal{O}(g(n))$ implies $g(n) = \mathcal{O}(f(n))$.\\

False. Counterexample:\\[6pt]
\indent \indent Let $f(n)$ be $n$.\\[6pt]
\indent \indent Let $g(n) = n^2$.\\[6pt]
\indent \indent There exists a constant $c, \hspace{2pt} \forall n \geq n_0$ such that $n \leq c n^2$ but there does not exist a constant $c, \forall n \geq n_0$ such that $n^2 \leq c n$.\\[6pt]
\indent \indent Thus, $ n = \mathcal{O}(n^2$) but $n^2 \neq \mathcal{O}(n)$. \\

\subsubsection*{Question b}

$f(n) + g(n) = \Theta(min(f(n),g(n)))$.\\


\indent \indent$f(n) + g(n) = \Theta(min(f(n),g(n)))$ implies that there exists a constant $c_1$ and a constant $c_2$, for all $n \geq n_0$ such that: \\[2pt]
\indent \indent$ c_1 \leq (min(f(n),g(n)) \leq f(n) + g(n) \leq c_2 \cdot min((f(n),g(n))$\\[6pt]
\indent False. Counterexample: \\[6pt]
\indent \indent Let $f(n) = n $ and $g(n) = 1$\\[6pt]
\indent \indent There is no $c_2$ such that  $n+1 \leq c_2 \cdot min(n,1)$\\[6pt]
\indent \indent So, $f(n) + g(n) \neq \Theta(min(f(n),g(n)))$\\[6pt]
\newpage
\subsubsection*{Question c}

$f(n) = \mathcal{O}(g(n))$ implies $lg(f(n)) = \mathcal{O}(lg(g(n)))$, where $lg(g(n)) \geq 1$ and $f(n) \geq 1$ for all sufficiently large $n$.\\

True: \\[6pt]
\indent We need to prove that $lg(f(n)=\mathcal{O}(lg(g(n)))$, or in other words $lg(f(n)) \leq d \cdot lg(g(n))$.\\[6pt]
\indent The first equation, as given: \\[6pt]
\indent \indent $f(n) = \mathcal{O}(g(n))$ implies that $f(n) \leq cg(n)$. \\[6pt]
\indent \indent Taking the log of both sides, we get: $lg(f(n) \leq lg(c) + lg(g(n))$\\[6pt]
\indent Now, set $d \cdot lg(g(n)) = lg(c)+lg(g(n))$\\[6pt]
\indent Divide both sides by $lg(g(n))$, we have $d= \frac{lg(c)}{lg(g)}+1$\\[6pt]
\indent Plugging this value of $d$ into the equation, we have: $lg(f(n)) \leq \left(\frac{lg(c)}{lg(g(n))}+1\right) \cdot lg(g(n))$\\[6pt]
\indent Since $lg(g(n))\geq 1, \hspace{3pt}lg(c) \geq \frac{lg(c)}{lg(g(n))}$, so clearly  $lg(f(n)) \leq (lg(c)+1) \cdot lg(g(n))$\\[6pt]
\indent And so, $lg(f(n)) \leq d \cdot lg(g(n))$, thus proving that $lg(f(n)) = \mathcal{O}(lg(g(n)))$.\\[6pt]

\subsubsection*{Question d}

$f(n) = \mathcal{O}(g(n))$ implies $2^{f(n)} = \mathcal{O}(2^{g(n)})$.\\

False. Counterexample: \\[6pt]
\indent Let $f(n) = 2n$ and $g(n) = n$. \\[6pt]
\indent So $2^{2n} = \mathcal{O}(2^{n})$ implies that $2^{2n} \leq c \cdot2^n$\\[6pt]
\indent \indent \indent \indent \indent \indent $2^{2n} \leq c \cdot2^n$ = $(2^2)^n \leq c \cdot 2^n$ = $4^n \leq c \cdot 2^n$\\[6pt]
\indent The value of $c$ is dependent on $n$, so there is no value of $c$ such that $c\geq 2n$ for all $n \geq n_0$. Hence, this statement is false.\\[6pt]

\subsubsection*{Question e}

$f(n) = \mathcal{O}((f(n))^2)$. \\

False.\\[6pt]
\indent \indent $f(n) = \mathcal{O}((f(n))^2)$ implies that $f(n) \leq c(f(n))^2$.\\[6pt]
\indent \indent But when $0 \leq f(n) \leq 1$, then there is no value of $c$ such that $f(n) \leq c \cdot (f(n))^2,  \forall n \geq n_0$.\\[6pt]
\indent Hence, this statement is false.\\

\newpage
\subsubsection*{Question f}

$f(n) = \mathcal{O}(g(n))$ implies $g(n) = \Omega(f(n))$.\\

True.\\
\indent \indent $f(n) = \mathcal{O}(g(n))$ implies that there exists a constant $c$ such that $f(n) \leq cg(n), \forall n \geq n_0$.\\[6pt]
\indent \indent Divide both sides by $c$:  $\frac{1}{c} f(n) \leq g(n) $\\[6pt]
\indent \indent Thus, $g(n) \geq \frac{1}{c} f(n)$, which means that $g(n)= \Omega(f(n))$.\\[6pt]

\subsubsection*{Question g}

$f(n) = \Theta(f(n/2))$.\\

\indent \indent $f(n) = \Theta(f(n/2))$ implies that there exists two constants such that:\\[6pt]
\indent \indent $c_1 \cdot f(n/2) \leq f(n) \leq c_2 \cdot f(n/2)$ \\[6pt]
\indent False. Counterexample:\\[6pt]
\indent \indent Let $f(n) = 2^n$ \\[6pt]
\indent \indent There is no $c$ such that: $2^n \leq c_2 \cdot 2^{n/2}$, for all of $n \geq n_0$. So this statement is false.\\[6pt]

\subsubsection*{Question h}

$f(n) + o(f(n)) = \Theta(f(n))$\\

True: \\

\indent \indent Let $g(n) = o(f(n)) $\\[6pt]
\indent \indent This implies that $g(n) \less f(n)$\\[6pt]
\indent \indent $f(n) + g(n) = \Theta(f(n))$ implies that there exists a constant $c_1$ and a constant $c_2$ such that $c_1 f(n) \leq f(n) + g(n) \leq c_2 f(n)$, for all $n \geq n_0$\\[6pt]
\indent \indent Let $c_1 = 1 $ and let $c_2 = 2$ \\[6pt]
\indent \indent It is true that $f(n) \leq f(n)+g(n) \leq  2(f(n))$\\[6pt]
\indent \indent Thus, it is true that $f(n) + o(f(n)) = \Theta(f(n))$.\\[6pt]

\end{document}