\documentclass[11pt]{article}
\usepackage[utf8]{inputenc}

% add LaTeX packages to use here
\usepackage{amsmath}
\usepackage{amssymb}
\usepackage{amsfonts}
\usepackage{amsthm}
\usepackage{fancyhdr}
\usepackage{lastpage}
\usepackage{enumitem}
\usepackage{framed}
\usepackage[most]{tcolorbox}
\usepackage{geometry}
\usepackage{graphicx}

 % set dimensions for page layout
\geometry
{
 left=5em,
 right=5em,
 bottom=5em,
 top=5.7em,
 headheight=110pt,
 showframe=false
}

\setlist[itemize]{leftmargin=*} % prevents indenting of itemize

% abbreviations for some common math symbols
\newcommand{\Rset}{\hbox{$\mathbb R$}}
\newcommand{\Nset}{\hbox{$\mathbb N$}}
\newcommand{\Pset}{\hbox{$\mathbb{N}^{+}$}}
\newcommand{\Zset}{\hbox{$\mathbb N$}}
\newcommand{\Qset}{\hbox{$\mathbb Q$}}

% theorem style
\newtheoremstyle{thmstyle}% name of the style to be used
  {0pt}% measure of space to leave above the theorem. E.g.: 3pt
  {0pt}% measure of space to leave below the theorem. E.g.: 3pt
  {}% name of font to use in the body of the theorem
  {}% measure of space to indent
  {\bfseries}% name of head font
  {.}% punctuation between head and body
  { }% space after theorem head; " " = normal inter-word space
  {}% Manually specify head

% theorem environment instance
\theoremstyle{thmstyle}
\newtheorem{theorem}{Theorem}

% shaded and framed solution environment 
\makeatletter
\newenvironment{shadedSolutionBox}
  {\setlength{\OuterFrameSep}{0in}%
  \definecolor{shadecolor}{gray}{.8}% shading of shaded solution box
  \bigskip%
  \@nameuse{shaded*}\par\noindent\ignorespaces \textit{Solution}.}
  {\hspace{\stretch{1}}\rule{1.5ex}{1.5ex}% adds filled box 
  \@nameuse{endshaded*}%
  \bigskip}
\makeatother

% shaded and framed theorem environment 
\makeatletter
\newenvironment{thm}
  {\setlength{\OuterFrameSep}{0in}%
  \definecolor{shadecolor}{gray}{1}% shading of shaded Theorem box
  \@nameuse{snugshade*}\par\noindent\ignorespaces%
   \@nameuse{theorem}}
  {\hspace{\stretch{1}}\scalebox{1.5}{\hbox{$\triangleleft$}}% adds triangle shape
  \@nameuse{endtheorem}%
  \@nameuse{endsnugshade*}%
  }
\makeatother

% header and footer elements of every page except the first.
\pagestyle{fancy}
\fancyfoot[L]{{\textsc{cisc} \small\selectfont 3220}}
\fancyhead[R]{{\small\selectfont\textsc{\studentLastName}}}
\fancyfoot[C]{{\small\selectfont\assignmentName}}
\fancyfoot[R]{{\small\selectfont\thepage\ of \pageref{LastPage}}}
\renewcommand{\headrulewidth}{0.8pt}
\renewcommand{\footrulewidth}{0.4pt}

% hline with variable thickness
\makeatletter
\def\thickhline{%
  \noalign{\ifnum0=`}\fi\hrule \@height \thickarrayrulewidth \futurelet
   \reserved@a\@xthickhline}
\def\@xthickhline{\ifx\reserved@a\thickhline
               \vskip\doublerulesep
               \vskip-\thickarrayrulewidth
             \fi
      \ifnum0=`{\fi}}
\makeatother

% length instance for \thickhline
\newlength{\thickarrayrulewidth} 
\setlength{\thickarrayrulewidth}{.8pt}

% header and footer for first page
\fancypagestyle{firstpage}
{
\fancyhf{}
\renewcommand{\footrulewidth}{0.4pt}
\renewcommand{\headrulewidth}{0pt}
\fancyhead[C]{%
\begin{tabular*}{\textwidth}{@{\extracolsep{\fill}}@{}l @{} c @{} r @{} }
{\small\selectfont\courseName}&{\normalsize\selectfont\assignmentName}&{\small\selectfont\studentFirstName\ \studentLastName}\\
\thickhline
&&{\scriptsize\selectfont\collaboratorNames}
\end{tabular*}%
}
\fancyfoot[R]{{\small\selectfont\thepage\ of \pageref{LastPage}}}
\fancyfoot[L]{{\footnotesize\selectfont\pdfcreationdate}}
}

\newcommand{\courseName}{Analysis of Algorithms} % course name

% your first name, your last name, and the assignment name
\newcommand{\studentLastName}{Friedman}
\newcommand{\studentFirstName}{Rachel} 
\newcommand{\assignmentName}{Assignment 6}
\newcommand{\collaboratorNames}{}




\begin{document} % marks the beginning of the document

\thispagestyle{firstpage} % institutes page style for first page

\setlength{\abovedisplayskip}{20pt} % space above math in align* environment
\setlength{\belowdisplayskip}{20pt} % space below math in align* environment



% marks the beginning of the document body

\begin{itemize}\setlength{\itemsep}{1em} % \itemsep is the spacing between items in environment
\item[1.]  Show that for every integer $n\in\Pset$: 
\begin{align*}
\sum^{n}_{i=1}\frac{1}{(4n+1)(4n-3)}\quad&=\quad\frac{n}{4n+1}.
\end{align*} 
\end{itemize}


\begin{shadedSolutionBox} Use mathematical induction to establish that for every integer $n\in\Pset$:
\begin{align*}
\sum^{n}_{i=1}\frac{1}{(4n+1)(4n-3)}\quad&=\quad\frac{n}{4n+1}.
\end{align*} 
\end{shadedSolutionBox}


\vspace{1in}

{\begin{flushright}{\Large\selectfont Continued on next page...}\end{flushright}}
\newpage

Here is a theorem:
\begin{thm} Let $a,b\in\Rset$.  For every integer $n\in\Nset$:
 \begin{align*}
\sum^{n}_{m=0}(a + mb)\quad&=\quad \frac{(n + 1)(2a + nb)}{2}.
 \end{align*}
\end{thm}

\begin{proof}  Let $a,b\in\Rset$. Observe that the identity in question readily follows from the well-known identity for evaluating the sum of the first $n$ natural numbers: 
\begin{align*}
\sum^{n}_{m=0}m\quad&=\quad \frac{n\bigl(n\,+\,1\bigr)}{2}.
\end{align*}
Indeed, for every $n\in\Nset$:
\begin{align*}
\sum^{n}_{m=0}\bigl(a\,+\,mb\bigr)\quad&=\quad a\sum^{n}_{m=0}1\;+\;b\sum^{n}_{m=0}m\\[1.2ex]
\quad&=\quad a\bigl(n\,+\,1\bigr)\;+\;b\,\frac{n\bigl(n\,+\,1\bigr)}{2}\\[1.2ex]
\quad&=\quad \frac{2a\bigl(n\,+\,1\bigr)\;+\;nb\bigl(n\,+\,1\bigr)}{2}\\[1.2ex]
\quad&=\quad\frac{\bigl(n\,+\,1\bigr)\bigl(2a\,+\,nb\bigr)}{2}.\\
\end{align*}
\end{proof}




\end{document}